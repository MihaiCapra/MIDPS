\section{Laboratory work implementation}

\subsection{Tasks and Points}
\begin{itemize}
	\item Basic Level (nota 5 || 6):
	
	\begin{itemize}
		\item Realizeaza un mini site cu 3 pagini statice
	\end{itemize}
	
	\item Normal Level (nota 7 || 8):
	
	\begin{itemize}
		\item ISite-ul trebuie sa pastreze toata informatia intr-o baza de date
	\end{itemize}
	\item Advanced Level (nota 9 || 10):
	
	\begin{itemize}
		\item Site-ul trebuie sa contina AJAX Requests.
    	\item Implimentarea XHR sau JSON responses. Careva din informatie trebuie sa fie dinamic incarcata pe pagina.
	\end{itemize}
\end{itemize}
    



\subsection{Analiza lucrarii de laborator}

Linkul la repozitorul Github:\\
\begin{center}
\url{https://github.com/MihaiCapra/MIDPS}
\end{center}

Am utilizat un host online pentru a tine webapp-ul si fisierele sale necesare.
\begin{center}
\includegraphics[scale=0.5]{images/host}
\end{center}
Pentru inceput am realizat trei fisiere html:pagina principala,portofoliu si mini\_store.Toate aceste pagini sunt interactive si isi schimba marimea dupa device-ul folosit datorita ca am utilizat foundation.zurb care la-m modificat dupa necesitatile mele.
\begin{flushleft}
\includegraphics[scale=0.5]{images/home}
\end{flushleft}
\begin{flushleft}
\includegraphics[scale=0.5]{images/portfolio}
\end{flushleft}
\begin{flushleft}
\includegraphics[scale=0.5]{images/mini_store}
\end{flushleft}
Dupa aceasta am configurat o baza de date cu componentele necesare(username,password,db\_name) utilizind cpanel.
Si am creat citeva fisiere PHP care raspund de conectarea si celecatrea basei de date,am utilizat si Ajax pentru obtinerea numarul paginei si afisarea posturilor necesare iar Pagination.php raspunde de afisarea limitata a posturilor pe pagina principala.
\begin{center}
\includegraphics[scale=0.6]{images/database_config}
\end{center}
\clearpage