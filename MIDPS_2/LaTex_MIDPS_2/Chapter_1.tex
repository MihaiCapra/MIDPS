\section{Realizarea lucrarii de laborator}

\subsection{Tasks and Points}

\begin{itemize}
\item Basic Level (nota 5 || 6) : 
\begin{itemize}

    \item conecteaza-te la server folosind SSH
    \item compileaza cel putin 2 sample programs din setul HelloWolrdPrograms folosind CLI
    \item executa primul commit folosind VCS

\end{itemize}
\item Normal Level (nota 7 || 8):
\begin{itemize}

   \item initializeaza un nou repositoriu
   \item configureaza-ti VCS
   \item crearea branch-urilor (creeaza cel putin 2 branches)
   \item commit pe ambele branch-uri (cel putin 1 commit per branch)

\end{itemize}
\item Advanced Level (nota 9 || 10): 
\begin{itemize}

   \item seteaza un branch to track a remote origin pe care vei putea sa faci push (ex. Github, Bitbucket or custom server)
   \item reseteaza un branch la commit-ul anterior
   \item merge 2 branches
   \item rezolvarea conflictelor a 2 branches

\end{itemize}
\end{itemize}

\subsection{Analiza lucrarii de laborator}
Linkul la repozitoriul GITHUB:
\begin{center}
\url{https://github.com/MihaiCapra/MIDPS}
\end{center}
Pentru a realiza aceasta lucrare de laborator $m-am$ inregistrat pe github.com si am instalat $git-bash$, am generat o cheie SSH si am adaugat aceasta cheie publica pe github pentru a identifica acest calculator.\\
\begin{center}
\includegraphics[scale=0.5]{images/github_ssh}
\end{center}
Pentru a compila programe scrise in $C++,Java,Python$ avem nevoie de a seta directiile spre $g++,javac$ si python in fisierul bash\_profile din directoriul unde este instalat $Git-Bash$.Pentru a compila programul scris in Java utilizam javac pentru compilare si java HelloGITHUB pentru a rula programul nostru,in cazul programuli $C++$ utilizam comanda $g++ $ hello.cpp -o hello, si ./hello pentru a rula programul si in cazul unui program scris in Python utilizam sintaxa python GITHUB.py pentru a rula programul.
\begin{center}
\includegraphics[scale=0.5]{images/java}
\includegraphics[scale=0.5]{images/cpp}\\
\end{center}
\begin{center}
\includegraphics[scale=1]{images/Python}
\end{center}
Pentru fiecare schimbare pe care o facem pe repozitoriu putem lasa un mesaj folosind comanda git commit -m "mesaj" astfel organizam mai bine repozitoriul si putem vedea  ce schimbari au avut loc.
\begin{center}
\includegraphics[scale=0.7]{images/git_commit}
\end{center}
Am initializat un nou repozitoriu cu numele NEWREPO cu git init,si am configurat acest repozitoriu cu git config \--global user.name si user.email.
\begin{center}
\includegraphics[scale=0.7]{images/git_init}\\
\includegraphics[scale=0.7]{images/repos}\\
\end{center}
Am creat doua branch-uri cu numele A si B folosind comanda git branch "numele".
\begin{center}
\includegraphics[scale=0.7]{images/git_branch}\\
\end{center}
Am adaugat un fisier pe branch-ul A.
\begin{center}
\includegraphics[scale=0.7]{images/A}
\end{center}
Am adaugat si un fisier pe branch-ul B.
\begin{center}
\includegraphics[scale=0.7]{images/B}
\end{center}
Cind accesam github.com ca master putem accepta schimbarile de pe celelalte branch-uri astfel fisierele vor fi adaugate pe master.La fel putem lasa si un comentariu pentru acel commit.
\begin{center}
\includegraphics[scale=0.7]{images/accept}\\
\includegraphics[scale=0.7]{images/accept_request}\\
\end{center}
Am setat branch-ul B track a remote.
\begin{center}
\includegraphics[scale=0.7]{images/track_remote}
\end{center}
Am resetat branch-ul B la un commit anterior.
\begin{center}
\includegraphics[scale=0.7]{images/git_reset}
\end{center}
Am facut merge la branch-ul B cu master.
\begin{center}
\includegraphics[scale=0.7]{images/git_merge_B}
\end{center}
In cazul cind pe un branch avem un fisier cu un continut oarecare si pe al branch acelasi fisier dar cu continut diferit atunci cind incerca sa facem merge a acestor doua branch-uri atunci primim un mesaj de conflict.Daca deschidem fisierul acolo vor fi afisate problemele care trebuie inlaturate.
\begin{center}
\includegraphics[scale=0.7]{images/conflict}\\
\includegraphics[scale=0.7]{images/problem}\\
\end{center}
Pentru a rezolva aceasta problema putem modifica continutul fisierului si dupa care faceem din nou git add si commit astfel rezolvam acest conflict. 


\clearpage